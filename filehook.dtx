% \iffalse meta-comment
%% Copyright (c) 2010 by Martin Scharrer <martin@scharrer-online.de>
%% -----------------------------------------------------------------
%%
%% This work may be distributed and/or modified under the
%% conditions of the LaTeX Project Public License, either version 1.3c
%% of this license or (at your option) any later version.
%% The latest version of this license is in
%%
%%   http://www.latex-project.org/lppl.txt
%%
%% and version 1.3c or later is part of all distributions of LaTeX
%% version 2008/05/04 or later.
%%
%% This work has the LPPL maintenance status `maintained'.
%%
%% The Current Maintainer of this work is Martin Scharrer.
%%
%% This work consists of the files filehook.dtx, filehook.ins
%% and the derived file filehook.sty.
%%
%% $Id: filehook.dtx 1968 2010-12-30 23:58:11Z martin $
% \fi
%
% \iffalse
%<*driver>
\ProvidesFile{filehook.dtx}
%</driver>
%<package>\NeedsTeXFormat{LaTeX2e}[1999/12/01]
%<package>\ProvidesPackage{filehook}
%<*package>
    [2010/12/30 v0.4 Hooks for input files]
%</package>
%
%<*driver>
\documentclass{ydoc}[2010/12/20]
\usepackage{filehook}[2010/12/30]
\usepackage{ifpdf}
\usepackage{hyperref}
\EnableCrossrefs
\CodelineIndex
\RecordChanges
\providecommand*\pkg{\texttt}
\listfiles
\begin{document}
  \DocInput{filehook.dtx}
  \PrintChanges
  \PrintIndex
\end{document}
%</driver>
% \fi
%
% \CheckSum{661}
%
% \CharacterTable
%  {Upper-case    \A\B\C\D\E\F\G\H\I\J\K\L\M\N\O\P\Q\R\S\T\U\V\W\X\Y\Z
%   Lower-case    \a\b\c\d\e\f\g\h\i\j\k\l\m\n\o\p\q\r\s\t\u\v\w\x\y\z
%   Digits        \0\1\2\3\4\5\6\7\8\9
%   Exclamation   \!     Double quote  \"     Hash (number) \#
%   Dollar        \$     Percent       \%     Ampersand     \&
%   Acute accent  \'     Left paren    \(     Right paren   \)
%   Asterisk      \*     Plus          \+     Comma         \,
%   Minus         \-     Point         \.     Solidus       \/
%   Colon         \:     Semicolon     \;     Less than     \<
%   Equals        \=     Greater than  \>     Question mark \?
%   Commercial at \@     Left bracket  \[     Backslash     \\
%   Right bracket \]     Circumflex    \^     Underscore    \_
%   Grave accent  \`     Left brace    \{     Vertical bar  \|
%   Right brace   \}     Tilde         \~}
%
%
% \changes{v0.1}{2010/04/08}{Initial version}
% \changes{v0.2}{2010/12/08}{Added support for 'memoir' class and 'scrlfile' package. Remove support for deprecated 'fink' package.}
% \changes{v0.3}{2010/12/20}{Added hooks for package and class files. Changed a warning to an error and added the 'force' option to overwrite this.}
% \changes{v0.4}{2010/12/30}{Improved support for 'memoir' class and 'scrlfile' package. Added AtBeginOfEveryFile and AtEndOfEveryFile hooks.
% Changed handling of macro arguments in hook code.
% }
%
% \GetFileInfo{filehook.dtx}
%
% \DoNotIndex{\newcommand,\newenvironment}
%
% \ifpdf
% \hypersetup{%
%   pdfauthor   = {Martin Scharrer <martin@scharrer-online.de>},
%   pdftitle    = {The filehook package},
%   pdfsubject  = {Documentation of LaTeX package 'filehook'},
%   pdfkeywords = {current filename, filename, file, name, LaTeX, TeX}
% }%
% \fi
% \clearpage
% \null
% \vspace*{-2em}
% \begin{center}
%   {\LARGE\sffamily The \emph{filehook} Package\\[\medskipamount]}
%   {\large Martin Scharrer \\[\medskipamount]\normalsize 
%   \url{martin@scharrer-online.de}\\[.8ex]
%   \url{http://www.ctan.org/pkg/filehook/}\\[\bigskipamount]}
%   {\large Version \fileversion\ -- \filedate}\\
% \end{center}
% \vspace{1.2em}%
%
% \begin{abstract}
% This package provides hooks for input files. Document and package authors can use these hooks to
% execute code at begin or the end of specific or all input files.
% \end{abstract}
%
%
% \section{Introduction}
% These package changes some internal \LaTeX{} macros used to load input files so that they include `hooks'.
% A hook is an (internal) macro executed at specific points. Normally it is initially empty, but can be extended using
% an user level macro. The most common hook in \LaTeX{} is the `At-Begin-Document' hook. Code can be added to this hook
% using \Macro\AtBeginDocument{<\TeX code>}.
%
%
% This package provides hooks for files read by the \LaTeX{} macros \Macro\input, \Macro\include and 
% \Macro\InputIfFileExists as well as (since v0.3 from 2010/12/20) for class and package files, 
% i.e. macros \Macro\documentclass, \Macro\LoadClassWithOptions and \Macro\LoadClass
% as well as \Macro\usepackage, \Macro\RequirePackageWithOptions and \Macro\RequirePackage.
% Note that \Macro\InputIfFileExists, and therefore its hooks, is used by the aforementioned macros.
% In v0.4 from 2010/12/30 special hooks where added which are executed for every read file, but will not be
% executed a second time by the internal \Macro\InputIfFileExists inside \Macro\input and \Macro\include.
%
% For all files a `AtBegin' and a `AtEnd' hook is installed. For \Macro\include files there is also a `After' hook which
% it is executed \emph{after} the page break (\Macro\clearpage) is inserted by the \Macro\include code.
% In contrast, the `AtEnd' hook is executed before the trailing page break 
% and the `AtBegin' hook is executed after the \emph{leading} page break.
% The `AtBegin' hook can be used to set macros to file specific values.
% These macros can be reset in the `AtEnd' hook to the parent file values.
% If these macros appear in the page header or footer they need to be reset `After' hook
% to ensure that the correct values are used for the last page.
%
% In addition to general hooks which are executed for all files of there type,
% file specific one can be defined which are only executed for the named file.
% The hooks for classes and packages are always specific to one file.
%
% Older versions of this package provided the file name as argument |#1| for the general hooks.
% This has been changed in v0.4 from 2010/12/30: the hook code is stored and executed without modifications,
% i.e.\ macro argument characters (|#|) are now handled like normal and don't have to be doubled.
% See section~\ref{sec:upgrade} for information how to upgrade older documents.
%
% \section{Usage}
% The below macros can be used to add material (\TeX{} code) to the related hooks.
% All `AtBegin' macros will \emph{append} the code to the hooks, but the `AtEnd' and `After' macros will \emph{prefix} the code instead.
% This ensures that two different packages adding material in `AtBegin'/`AtEnd' pairs do not overlap each other.
% Instead the later used package adds the code closer to the file content, `inside' the material added by the first package.
% Therefore it is safely possible to surround the content of a file with multiple \LaTeX{} environments using multiple `AtBegin'/`AtEnd' macro calls.
% If required inside another package a different order can be enforced by using the internal hook macros shown in the implementation section.
%
%
%^^A Some internal macros to draw the hook positions:
% \def\Hook#1{\textsf{Hook: #1}\MacroArgs}
%
% \def\DrawInputIfB#1{%
% \hbox{\vbox{%
%   \sffamily
%   \hbox{\Macro\InputIfFileExists:}%
%   \hbox{\fbox{\vbox{%
%     \hbox{\Hook{AtBeginOfFile}{<file name>}}%
%     #1%
%     \hbox{\Hook{AtBeginOfFiles}}%
%     \hbox{\fbox{\emph{Content}}}%
%     \vspace{2pt}%
%     \hbox{\Hook{AtEndOfFiles}}%
%     \hbox{\Hook{AtEndOfFile}{<file name>}}%
%   }}}%
% }}%
% }%
% 
% \def\DrawInputIf#1{%
% \hbox{\vbox{%
%   \sffamily
%   \hbox{\Macro\InputIfFileExists:}%
%   \hbox{\fbox{\vbox{%
%     \hbox{\Hook{AtBeginOfEveryFile}}%
%     \hbox{\Hook{AtBeginOfFile}{<file name>}}%
%     #1%
%     \hbox{\Hook{AtBeginOfFiles}}%
%     \hbox{\fbox{\emph{Content}}}%
%     \vspace{2pt}%
%     \hbox{\Hook{AtEndOfFiles}}%
%     \hbox{\Hook{AtEndOfFile}{<file name>}}%
%     \hbox{\Hook{AtEndOfEveryFile}}%
%   }}}%
% }}%
% }%
%
%
% \subsection*{Every File}
%
% \DescribeMacro{\AtBeginOfEveryFile}{<\TeX\ code>}
% \DescribeMacro{\AtEndOfEveryFile}{<\TeX\ code>}
% Sometime certain code should be executed at the begin and end of every read file, e.g.\ pushing and popping a file stack.
% The \Macro\At...OfFiles hooks already do a good job here. Unfortunately there is the issue with the \Macro\clearpage in \Macro\include.
% The \Macro\AtEndOfFiles is executed before it, which can cause issues with page headers and footers.
% A workaround, e.g.\ done by older versions of the \pkg{currfile} package, is to execute the code twice for include files:
% once in the |include| related hooks and once in the |OfFiles| hooks.
%
% A better solution for this problem was added in v0.4 from 2010/12/30:
% the |EveryFile| hooks will be executed exactly once for every file, independent if it is
% read using \Macro\input, \Macro\include or \Macro\InputIfFileExists.
% Special care is taken to suppress them for the \Macro\InputIfFileExists inside \Macro\input and \Macro\include.
%
% These hooks are located around the more specific hooks:
% For \Macro\input files the `Begin' hook is executed before the \Macro\AtBeginOfInputs hook and the `End' hook after
% the \Macro\AtEndOfInputs.
% Similarly, for \Macro\include files the `Begin' hook is executed before the \Macro\AtBeginOfIncludes hook and the `End' hook after
% the \Macro\AfterIncludes (!).
% For files read by \Macro\InputIfFileExists (e.g. also for \Macro\usepackage, etc.) they are executed before and after the
% \Macro\AtBeginOfFiles and \Macro\AtEndOfFiles hooks, respectively.
% Note that the \Macro\AtBeginOfEveryFile hook is executed before the \Macro\AtBeginOfPackageFile/\Macro\AtBeginOfClassFile hooks
% and that the \Macro\AtEndOfEveryFile hook is executed also before the hooks \Macro\AtEndOfPackageFile/\Macro\AtEndOfClassFile.
% Therefore the `Every' and `PackageFile'/`ClassFile' hooks do not nest correctly like all other hooks do.
%
%
% \subsection*{All Files}
%
% \DescribeMacro{\AtBeginOfFiles}{<\TeX\ code>}
% \DescribeMacro{\AtEndOfFiles}{<\TeX\ code>}
% These macros add the given \marg{code} to two hooks executed for all files read using the \Macro\InputIfFileExists macro. This macro is used internally by the 
% \Macro\input, \Macro\include and \Macro\usepackage/\Macro\RequirePackage macros. Packages and classes might use it to include additional or auxiliary files.
% Authors can exclude those files from the hooks by using \Macro\IfFileExists{<file name>}|{\@input\@filef@und}{}| instead.
%
% \DescribeMacro{\AtBeginOfFile}{<file name with extension>}{<\TeX\ code>}
% \DescribeMacro{\AtEndOfFile}{<file name with extension>}{<\TeX\ code>}
% Like the {\expandafter\Macro\csname...OfIncludeFile\endcsname{<file name>}{<\TeX\ code>}} macros above, just for `all' read files. Here the \meta{file name} should include the file
% extension! The \meta{code} must not include any macro arguments (|#1|).
%
% The `all files' hooks are closer to the file content than the \Macro\input and \Macro\include hook, i.e.\ the \Macro\AtBeginOfFiles comes \emph{after} the \Macro\AtBeginOfIncludes and
% the \Macro\AtEndOfFiles comes \emph{before} the \Macro\AtEndOfIncludes hook.
%
% The following figure shows the positions of the hooks inside the macro:\par\medskip
% \centerline{\DrawInputIf{}}
%
%
% \subsection*{Include Files}
% \DescribeMacro{\AtBeginOfIncludes}{<\TeX\ code>}
% \DescribeMacro{\AtEndOfIncludes}{<\TeX\ code>}
% \DescribeMacro{\AfterIncludes}{<\TeX\ code>}
% All these macros take one argument (some \TeX{} code) which is added to the specific hook for files read using \Macro\include.
% The code can use the macro argument |#1|
% which will be expanded to the include \meta{file name}, i.e.\ the hooks are macros with one argument which will be the file name.
% As described above the `AtEnd' hook is executed before and the `After' hook is executed after the trailing \Macro\clearpage.
% Note that material which appears in the page header or footer should be updated in the `After' hook, not the `AtEnd` hook, to ensure
% that the old values are still valid for the last page.
%
% \DescribeMacro{\AtBeginOfIncludeFile}{<file name>}{<\TeX\ code>}
% \DescribeMacro{\AtEndOfIncludeFile}{<file name>}{<\TeX\ code>}
% \DescribeMacro{\AfterIncludeFile}{<file name>}{<\TeX\ code>}
% These file-specific macros take the two arguments. The \meta{code} is only executed for the file with the given \meta{file name}
% and only if it is read using \Macro\include. It is not allowed to use macro arguments inside the code.
% The \meta{file name} should be identical to the name used for \Macro\include and not include the `|.tex|' extension.
%
% The following figure shows the positions of the hooks inside the macro:\par\medskip
% \centerline{\hbox{\vbox{%
%   \sffamily
%   \hbox{\Macro\include:}%
%   \hbox{\fbox{\vbox{%
%    \hbox{\Macro\clearpage~~(implicit)}%
%    \hbox{\Hook{AtBeginOfEveryFile}}%
%    \hbox{\Hook{AtBeginOfIncludeFile}{<file name>}}%
%    \hbox{\Hook{AtBeginOfIncludes}}%
%    \hbox{\fbox{\DrawInputIfB{}}}%
%    \vspace{2pt}%
%    \hbox{\Hook{AtEndOfIncludes}}%
%    \hbox{\Hook{AtEndOfIncludeFile}{<file name>}}%
%    \hbox{\Macro\clearpage~~(implicit)}%
%    \hbox{\Hook{AfterIncludes}}%
%    \hbox{\Hook{AfterIncludeFile}{<file name>}}%
%    \hbox{\Hook{AtEndOfEveryFile}}%
%   }}}%
% }}}
%
%
% \subsection*{Input Files}
%
% \DescribeMacro{\AtBeginOfInputs}{<\TeX\ code>}
% \DescribeMacro{\AtEndOfInputs}{<\TeX\ code>}
% Like the \Macro\...OfIncludes{code} macros above, just for file read using \Macro\input. Again, the macro argument |#1| can be used inside the \meta{code} and
% will be expanded to the \meta{file name}.
%
% \DescribeMacro{\AtBeginOfInputFile}{<file name>}{<\TeX\ code>}
% \DescribeMacro{\AtEndOfInputFile}{<file name>}{<\TeX\ code>}
% Like the \Macro\...OfIncludeFile{<file name>}{code} macros above, just for file read using \Macro\input. Here the \meta{file name} should include the file
% extension! The \meta{code} must not include any macro arguments (|#1|).
%
% The following figure shows the positions of the hooks inside the macro:\par\medskip
% \centerline{\hbox{\vbox{%
%   \sffamily
%   \hbox{\Macro\input:}%
%   \hbox{\fbox{\vbox{%
%    \hbox{\Hook{AtBeginOfEveryFile}}%
%    \hbox{\Hook{AtBeginOfInputFile}{<file name>}}%
%    \hbox{\Hook{AtBeginOfInputs}}%
%    \hbox{\fbox{\DrawInputIfB{}}}%
%    \vspace{2pt}%
%    \hbox{\Hook{AtEndOfInputs}}%
%    \hbox{\Hook{AtEndOfInputFile}{<file name>}}%
%    \hbox{\Hook{AtEndOfEveryFile}}%
%   }}}%
% }}}
%
%
%
% \subsection*{Package Files}
%
% \DescribeMacro{\AtBeginOfPackageFile}{<package name>}{<\TeX\ code>}
% \DescribeMacro{\AtEndOfPackageFile}*{<package name>}{<\TeX\ code>}
% This macros install the given \MacroArgs<\TeX\ code> in the `AtBegin' and `AtEnd' hooks of the given package file.
% The \Macro\AtBeginOfPackageFile simply executes \Macro\AtBeginOfFile{<package name>.sty}{<\TeX code>}.
% Special care is taken to ensure that the `AtEnd' code is executed \emph{after} any code installed by the package itself
% using the \LaTeX\ macro \Macro\AtEndOfPackage.
% If the starred version is used and the package is already loaded the code
% is executed right away.
%
% The following figure shows the positions of the hooks inside the macros:\par\medskip
% \centerline{\hbox{\vbox{%
%   \sffamily
%   \hbox{\Macro\usepackage/\Macro\RequirePackage/\Macro\RequirePackageWithOptions:}%
%   \hbox{\fbox{\vbox{%
%    \hbox{\fbox{\DrawInputIf{\hbox{ (includes AtBeginOfPackageFile\MacroArgs{<file name>})}}}}%
%    \vspace{2pt}%
%    \hbox{\Hook{AtEndOfPackage}~~(\LaTeX\ hook)}%
%    \hbox{\Hook{AtEndOfPackageFile}{<file name>}}%
%   }}}%
% }}}
%
% \subsection*{Class Files}
%
% \DescribeMacro{\AtBeginOfClassFile}{<class name>}{<\TeX\ code>}
% \DescribeMacro{\AtEndOfClassFile}*{<class name>}{<\TeX\ code>}
% This macros install the given \MacroArgs<\TeX\ code> in the `AtBegin' and `AtEnd' hooks of the given class file.
% They work with classes loaded using \Macro\LoadClass, \Macro\LoadClassWithOptions and also \Macro\documentclass.
% However, in the latter case |filehook| must be loaded using \Macro\RequirePackage beforehand.
% The macro \Macro\AtBeginOfClassFile simply executes \Macro\AtBeginOfFile{<class name>.cls}{\ldots}.
% Special care is taken to ensure that the `AtEnd' code is executed \emph{after} any code installed by the class itself
% using the \LaTeX\ macro \Macro\AtEndOfClass.
% If the starred version is used and the class is already loaded the code
% is executed right away.
%
% The following figure shows the positions of the hooks inside the macros:\par\medskip
% \centerline{\hbox{\vbox{%
%   \sffamily
%   \hbox{\Macro\documentclass/\Macro\LoadClass/\Macro\LoadClassWithOptions:}%
%   \hbox{\fbox{\vbox{%
%    \hbox{\fbox{\DrawInputIf{\hbox{ (includes AtBeginOfClassFile\MacroArgs{<file name>})}}}}%
%    \vspace{2pt}%
%    \hbox{\Hook{AtEndOfClass}~~(\LaTeX\ hook)}%
%    \hbox{\Hook{AtEndOfClassFile}{<file name>}}%
%   }}}%
% }}}
%
%
%
% \section{Compatibility Issues with other Packages}
% The |filehook| package might clash with other packages or classes which also redefine \Macro\InputIfFileExists.
% Special compatibility code is in place for the known packages listed below (in their current implementation).
% If any other unknown definition is found an error will be raised. The package option `|force|' can be used
% to prevent this and to force the redefinition of this macro.
% Then any previous modifications will be lost, which will most likely break the other package.
% Please do not hesitate to inform the author of |filehook| of any encountered problems with other packages.
%
%
% \subsection*{jmlrbook}
% The |jmlrbook| class from the |jmlr| bundle temporary redefines \Macro\InputIfFileExists to import papers.
% The `original' definition is saved away at load time of the package and is used internally by the new definition.
% This means that the hooks will not be active for this imported files because |filehook| is loaded after the class.
% This should not affect its normal usage.
% Note that, in theory, the package could be loaded before \Macro\documentclass using \Macro\RequirePackage to enable the file hooks also for these files.
%
%
% \subsection*{memoir}
% The |memoir| class redefines \Macro\InputIfFileExists to add own hooks identical to the |At...OfFiles| hooks (there called \Macro\AtBeginFile and \Macro\AtEndFile).
% This hooks will be moved to the corresponding ones of |filehook| and will keep working as normal.
% Since v0.4 from 2010/12/30 this modification will be also applied when the |filehook| package is loaded (using \Macro\RequirePackage) \emph{before} the
% |memoir| class. However, the hooks from |filehook| need to be temporally disabled while reading the |memoir| class.
% They will not be triggered for all files read directly by this class, like configuration and patch files.
% Note that the `At...OfClassFile' hooks still work for the |memoir| class file itself. In fact they are used to restore the default definition of \Macro\InputIfFileExists
% at the begin and patch it at the end of the class file.
%
%
% \subsection*{scrlfile}
% The |scrlfile| package from the \emph{koma-script} bundle redefines \Macro\InputIfFileExists to allow file name aliases and to also add hooks.
% If required it should be loaded before |filehook|, which will add its hooks correctly to the modified definition.
% Since v0.4 from 2010/12/30 this modification will be also applied when the |scrlfile| package is loaded after |filehook|.
%
%
% \subsection*{fink}
% The |filehook| and |currfile| packages where written as replacements for the |fink| package, where |filehook| provides the necessary hooks for |currfile|.
% The |fink| package has now been deprecated in favour of |currfile| and should not be used anymore. The |fink| compatibility code has been removed from |filehook|
% and both cannot be used successfully together as both redefine the \Macro\InputIfFileExists macro.
%
%
% \subsection*{listings}
% The \pkg{listings} package uses \Macro\input inside \Macro\lstinputlisting. Therefore the |InputFile|(|s|) and |File|(|s|) hooks are also triggered for these files.
% Please note that this hooks are executing inside a verbatim environment. While the code in the hook is not affected (because it was added outside the verbatim
% environment), any further code read using any input macro (\Macro\input, \Macro\@input, \Macro\@@input (\TeX's \Macro\input), \ldots) will be processed verbatim and typeset
% as part of the listing. A known package suffering from this is \pkg{svn-multi} which loads |.svx| files for every |.tex| file.
% A workaround for this issue is to temporally redefine \Macro\input to \Macro\@input for \Macro\lstinputlisting:\\
% |   {\let\input\@input\lstinputlisting{...}}|.
%
% \section{Upgrade Guide}
% \label{sec:upgrade}
% This sections gives information for users of older versions of this package which unfortunately might not be 100\% backwards compatible.
%
% \subsection*{Upgrade to v0.4 - 2010/12/30}
% \begin{itemize}
%   \item The macro \Macro\AfterIncludeFile was misspelled as \Macro\AfterOfIncludeFile in the implementation of earlier versions, but not in the documentation.
%   This has now be corrected. Please adjust your code to use the correct name and to require the |filehook| package from 2010/12/30.
%   \item All general hooks (the one not taking a file argument) used to have an implicit argument |#1| which was expanded to the file name (i.e.\ the argument of \Macro\input etc.).
%    This has now be changed, so that macro arguments are not handled special in hook code. Older hooks which use this argument should either use \Macro\filehookfilearg\relax
%    (identical to the old |#1|) or \Macro\filehookfilename (parsed argument, always with file extension). These macros are available in \emph{all} hooks.
% \end{itemize}
%
% \StopEventually{}
% \clearpage
% \section{Implementation}\label{sec:impl}
%
% \iffalse
%<*package>
% \fi
%
% \subsection{Options}
%    \begin{macrocode}
\newif\iffilehook@force
\DeclareOption{force}{\filehook@forcetrue}
\ProcessOptions\relax
%    \end{macrocode}
%
%
%
% \subsection{Initialisation of Hooks}
% The general hooks are initialised to call the file specific hooks.
%
% \begin{macro}{\filehook@include@atbegin}
%    \begin{macrocode}
\def\filehook@include@atbegin{%
  \let\InputIfFileExists\filehook@@InputIfFileExists
  \filehook@every@atbegin
  \@nameuse{\filehook@include@atbegin@\filehookfilearg}%
}
%    \end{macrocode}
% \end{macro}
%
% \begin{macro}{\filehook@include@atend}
%    \begin{macrocode}
\def\filehook@include@atend{%
  \@nameuse{\filehook@include@atend@\filehookfilearg}%
}
%    \end{macrocode}
% \end{macro}
%
% \begin{macro}{\filehook@include@after}
%    \begin{macrocode}
\def\filehook@include@after{%
  \@nameuse{\filehook@include@after@\filehookfilearg}%
  \filehook@every@atend
}
%    \end{macrocode}
% \end{macro}
%
% \begin{macro}{\filehook@input@atbegin}
%    \begin{macrocode}
\def\filehook@input@atbegin{%
  \let\InputIfFileExists\filehook@@InputIfFileExists
  \filehook@every@atbegin
  \@nameuse{\filehook@input@atbegin@\filehookfilename}%
  % New stuff is added here
}
%    \end{macrocode}
% \end{macro}
%
% \begin{macro}{\filehook@input@atend}
%    \begin{macrocode}
\def\filehook@input@atend{%
  % New stuff is added here
  \@nameuse{\filehook@input@atend@\filehookfilename}%
  \filehook@every@atend
}
%    \end{macrocode}
% \end{macro}
%
% \begin{macro}{\filehook@atbegin}
%    \begin{macrocode}
\def\filehook@atbegin{%
  \@nameuse{\filehook@atbegin@\filehookfilename}%
  % New stuff is added here
}
%    \end{macrocode}
% \end{macro}
%
% \begin{macro}{\filehook@atend}
%    \begin{macrocode}
\def\filehook@atend{%
  % New stuff is added here
  \@nameuse{\filehook@atend@\filehookfilename}%
}
%    \end{macrocode}
% \end{macro}
%
% \begin{macro}{\filehook@every@atbegin}
%    \begin{macrocode}
\def\filehook@every@atbegin{}
%    \end{macrocode}
% \end{macro}
%
% \begin{macro}{\filehook@every@atend}
%    \begin{macrocode}
\def\filehook@every@atend{}
%    \end{macrocode}
% \end{macro}
%
%
%
% \subsection{Hook Modification Macros}
% The following macros are used to modify the hooks, i.e.\ to prefix or append code to them.
%
%
% \subsubsection*{Internal Macros}
%
% The macro prefixes for the file specific hooks are stored in macros to reduce the number of 
% tokens in the following macro definitions.
%    \begin{macrocode}
\def\filehook@include@atbegin@{filehook@include@atbegin@}
\def\filehook@include@atend@{filehook@include@atend@}
\def\filehook@include@after@{filehook@include@after@}
\def\filehook@input@atbegin@{filehook@input@atbegin@}
\def\filehook@input@atend@{filehook@input@atend@}
\def\filehook@input@after@{filehook@input@after@}
\def\filehook@atbegin@{filehook@atbegin@}
\def\filehook@atend@{filehook@atend@}
\def\filehook@after@{filehook@after@}
%    \end{macrocode}
%
%
% \begin{macro}{\filehook@append}
% Uses default \LaTeX{} macro.
%    \begin{macrocode}
\def\filehook@append{\g@addto@macro}
%    \end{macrocode}
% \end{macro}
%
% \begin{macro}{\filehook@prefix}
% Prefixes code without an argument to a hook.
%    \begin{macrocode}
\long\def\filehook@prefix#1#2{%
  \begingroup
    \@temptokena{#2}%
    \toks@\expandafter{#1}%
    \xdef#1{\the\@temptokena\the\toks@}%
  \endgroup
}
%    \end{macrocode}
% \end{macro}
%
%
% \begin{macro}{\filehook@parsefile}
% Parses filename using the \LaTeX\ macro and defines it with an extension added if required.
%    \begin{macrocode}
\def\filehook@parsefile#1{%
  \edef\filehookfilearg{#1}%
  \filename@parse{#1}%
  \edef\filehookfilename{\filename@area\filename@base.\ifx\filename@ext\relax tex\else\filename@ext\fi}%
}
%    \end{macrocode}
% \end{macro}
%
%
% \subsubsection*{User Level Macros}
% The user level macros simple use the above defined macros on the appropriate hook.
%
% \begin{macro}{\AtBeginOfIncludes}
%    \begin{macrocode}
\newcommand*\AtBeginOfIncludes{%
  \filehook@append\filehook@include@atbegin
}
%    \end{macrocode}
% \end{macro}
%
% \begin{macro}{\AtEndOfIncludes}
%    \begin{macrocode}
\newcommand*\AtEndOfIncludes{%
  \filehook@prefix\filehook@include@atend
}
%    \end{macrocode}
% \end{macro}
%
% \begin{macro}{\AfterIncludes}
%    \begin{macrocode}
\newcommand*\AfterIncludes{%
  \filehook@prefix\filehook@include@after
}
%    \end{macrocode}
% \end{macro}
%
% \begin{macro}{\AtBeginOfIncludeFile}
%    \begin{macrocode}
\newcommand*\AtBeginOfIncludeFile[1]{%
  \@ifundefined{\filehook@include@atbegin@#1.tex}%
    {\global\@namedef{\filehook@include@atbegin@#1.tex}{}}%
    {}%
  \expandafter\filehook@append\csname\filehook@include@atbegin@#1.tex\endcsname%
}
%    \end{macrocode}
% \end{macro}
%
% \begin{macro}{\AtEndOfIncludeFile}
%    \begin{macrocode}
\newcommand*\AtEndOfIncludeFile[1]{%
  \@ifundefined{\filehook@include@atend@#1.tex}%
    {\global\@namedef{\filehook@include@atend@#1.tex}{}}%
    {}%
  \expandafter\filehook@prefix\csname\filehook@include@atend@#1.tex\endcsname
}
%    \end{macrocode}
% \end{macro}
%
% \begin{macro}{\AfterIncludeFile}
%    \begin{macrocode}
\newcommand*\AfterIncludeFile[1]{%
  \@ifundefined{\filehook@include@after@#1.tex}%
    {\long\global\@namedef{\filehook@include@after@#1.tex}{}}%
    {}%
  \expandafter\filehook@prefix\csname\filehook@include@after@#1.tex\endcsname
}
%    \end{macrocode}
% \end{macro}
%
% \begin{macro}{\AtBeginOfInputs}
%    \begin{macrocode}
\newcommand*\AtBeginOfInputs{%
  \filehook@append\filehook@input@atbegin
}
%    \end{macrocode}
% \end{macro}
%
% \begin{macro}{\AtEndOfInputs}
%    \begin{macrocode}
\newcommand*\AtEndOfInputs{%
  \filehook@prefix\filehook@input@atend
}
%    \end{macrocode}
% \end{macro}
%
% \begin{macro}{\AtBeginOfInputFile}
%    \begin{macrocode}
\newcommand*\AtBeginOfInputFile[1]{%
  \filehook@parsefile{#1}%
  \@ifundefined{\filehook@input@atbegin@\filehookfilename}%
    {\global\@namedef{\filehook@input@atbegin@\filehookfilename}{}}%
    {}%
  \expandafter\filehook@append\csname\filehook@input@atbegin@\filehookfilename\endcsname
}
%    \end{macrocode}
% \end{macro}
%
% \begin{macro}{\AtEndOfInputFile}
%    \begin{macrocode}
\newcommand*\AtEndOfInputFile[1]{%
  \filehook@parsefile{#1}%
  \@ifundefined{\filehook@input@atend@\filehookfilename}%
    {\global\@namedef{\filehook@input@atend@\filehookfilename}{}}%
    {}%
  \expandafter\filehook@prefix\csname\filehook@input@atend@\filehookfilename\endcsname%
}
%    \end{macrocode}
% \end{macro}
%
% \begin{macro}{\AtBeginOfFiles}
%    \begin{macrocode}
\newcommand*\AtBeginOfFiles{%
  \filehook@append\filehook@atbegin
}
%    \end{macrocode}
% \end{macro}
%
% \begin{macro}{\AtEndOfFiles}
%    \begin{macrocode}
\newcommand*\AtEndOfFiles{%
  \filehook@prefix\filehook@atend
}
%    \end{macrocode}
% \end{macro}
%
% \begin{macro}{\AtBeginOfEveryFile}
%    \begin{macrocode}
\newcommand*\AtBeginOfEveryFile{%
  \filehook@append\filehook@every@atbegin
}
%    \end{macrocode}
% \end{macro}
%
% \begin{macro}{\AtEndOfEveryFile}
%    \begin{macrocode}
\newcommand*\AtEndOfEveryFile{%
  \filehook@prefix\filehook@every@atend
}
%    \end{macrocode}
% \end{macro}
%
% \begin{macro}{\AtBeginOfFile}
%    \begin{macrocode}
\newcommand*\AtBeginOfFile[1]{%
  \filehook@parsefile{#1}%
  \@ifundefined{\filehook@atbegin@\filehookfilename}%
    {\global\@namedef{\filehook@atbegin@\filehookfilename}{}}%
    {}%
  \expandafter\filehook@append\csname\filehook@atbegin@\filehookfilename\endcsname
}
%    \end{macrocode}
% \end{macro}
%
% \begin{macro}{\AtEndOfFile}
%    \begin{macrocode}
\newcommand*\AtEndOfFile[1]{%
  \filehook@parsefile{#1}%
  \@ifundefined{\filehook@atend@\filehookfilename}%
    {\global\@namedef{\filehook@atend@\filehookfilename}{}}%
    {}%
  \expandafter\filehook@prefix\csname\filehook@atend@\filehookfilename\endcsname
}
%    \end{macrocode}
% \end{macro}
%
%
% \begin{macro}{\AtBeginOfPackageFile}[1]{package name}
% Simply add the package extension and calls the general macro.
%    \begin{macrocode}
\newcommand*\AtBeginOfPackageFile[1]{%
    \AtBeginOfFile{#1.\@pkgextension}%
}
%    \end{macrocode}
% \end{macro}
%
%
% \begin{macro}{\AtEndOfPackageFile}
%    \begin{macrocode}
\newcommand*\AtEndOfPackageFile{%
    \@ifnextchar*\AtEndOfPackageFile@star\AtEndOfPackageFile@normal
}
%    \end{macrocode}
% \end{macro}
%
% \begin{macro}{\AtEndOfPackageFile@star}[2]{package name}{code}
% If the package is already loaded the code is executed right away.
% Otherwise it is installed normally.
%    \begin{macrocode}
\def\AtEndOfPackageFile@star*#1#2{%
    \@ifpackageloaded{#1}%
        {#2}%
        {\AtEndOfPackageFile@normal{#1}{#2}}%
}
%    \end{macrocode}
% \end{macro}
%
% \begin{macro}{\AtEndOfPackageFile@normal}[2]{package name}{code}
% Installs the code at the end of the package file inside a \Macro\AtEndOfPackage
% command to ensure it is executed after any \Macro\AtEndOfPackage code installed
% by the package itself.
%
% Note if the package was already loaded or is not loaded at all the installed
% code is never executed.
%    \begin{macrocode}
\def\AtEndOfPackageFile@normal#1#2{%
    \AtEndOfFile{#1.\@pkgextension}{\AtEndOfPackage{#2}}%
}
%    \end{macrocode}
% \end{macro}
%
%
% \begin{macro}{\AtBeginOfClassFile}[1]{class name}
% Simply add the class extension and calls the general macro.
%    \begin{macrocode}
\newcommand*\AtBeginOfClassFile[1]{%
    \AtBeginOfFile{#1.\@clsextension}%
}
%    \end{macrocode}
% \end{macro}
%
%
% \begin{macro}{\AtEndOfClassFile}[2]{class name}{code}
%    \begin{macrocode}
\newcommand*\AtEndOfClassFile{%
    \@ifnextchar*\AtEndOfClassFile@star\AtEndOfClassFile@normal
}
%    \end{macrocode}
% \end{macro}
%
% \begin{macro}{\AtEndOfClassFile@star}[2]{class name}{code}
% If the class is already loaded the code is executed right away.
% Otherwise it is installed normally.
%    \begin{macrocode}
\def\AtEndOfClassFile@star*#1#2{%
    \@ifclassloaded{#1}%
        {#2}%
        {\AtEndOfClassFile@normal{#1}{#2}}%
}
%    \end{macrocode}
% \end{macro}
%
% \begin{macro}{\AtEndOfClassFile@normal}[2]{class name}{code}
% Installs the code at the end of the class file inside a \Macro\AtEndOfClass
% command to ensure it is executed after any \Macro\AtEndOfClass code installed
% by the class itself.
%
% Note if the class was already loaded or is not loaded at all the installed
% code is never executed.
%    \begin{macrocode}
\def\AtEndOfClassFile@normal#1#2{%
    \AtEndOfFile{#1.\@clsextension}{\AtEndOfClass{#2}}%
}
%    \end{macrocode}
% \end{macro}
%
%
% \subsection{Installation of Hooks}
% The \Macro\@input@ and \Macro\@iinput macros from |latex.ltx| are redefined to install the hooks.
%
% First the original definitions are saved away.
% \begin{macro}{\filehook@orig@@input@}
%    \begin{macrocode}
\let\filehook@orig@@input@\@input@
%    \end{macrocode}
% \end{macro}
%
% \begin{macro}{\filehook@orig@@iinput}
%    \begin{macrocode}
\let\filehook@orig@@iinput\@iinput
%    \end{macrocode}
% \end{macro}
%
% \begin{macro}{\@input@}
% This macro is redefined for the \Macro\include file hooks.
% Checks if the next command is \Macro\clearpage which indicates that we are inside \Macro\@include.
% If so the hooks are installed, otherwise the original macro is used unchanged.
% For the `after' hook an own \Macro\clearpage is inserted and the original one is gobbled.
%
%    \begin{macrocode}
\def\@input@#1{%
  \@ifnextchar\clearpage
    {%
     \filehook@parsefile{#1}%
     \filehook@include@atbegin
     \filehook@orig@@input@{#1}%
     \filehook@parsefile{#1}%
     \filehook@include@atend
     \clearpage
     \filehook@include@after
     \@gobble
    }%
    {\filehook@orig@@input@{#1}}%
}
%    \end{macrocode}
% \end{macro}
%
% \begin{macro}{\@iinput}
% This macro is redefined for the \Macro\input file hooks.
% it simply surrounds the original macro with the hooks.
%    \begin{macrocode}
\def\filehook@@iinput#1{%
  \filehook@parsefile{#1}%
  \filehook@input@atbegin
  \filehook@orig@@iinput{#1}%
  \filehook@parsefile{#1}%
  \filehook@input@atend
}
\let\@iinput\filehook@@iinput
%    \end{macrocode}
% \end{macro}
%
%
%
% The |filehook| default definition of \Macro\InputIfFileExists is defined here
% together with alternatives definitions for comparison.
% There are stored first in a token register and later stored in a macro which is expanded if required.
% This is always done inside a group to keep them temporary only.
% The token register is used to avoid doubling of macro argument characters.
%
%    \begin{macrocode}
\begingroup
\toks@={%}
%    \end{macrocode}
%
% \begin{macro}{\latex@InputIfFileExists}
%    \begin{macrocode}
\long\def\latex@InputIfFileExists#1#2{%
  \IfFileExists{#1}%
    {#2\@addtofilelist{#1}%
     \@@input\@filef@und
    }%
}
%    \end{macrocode}
% \end{macro}
%
%
% \begin{macro}{\filehook@default@InputIfFileExists}
%    \begin{macrocode}
\long\gdef\filehook@default@InputIfFileExists#1#2{%
  \IfFileExists{#1}%
    {#2\@addtofilelist{#1}%
     \filehook@parsefile{#1}%
     \filehook@every@atbegin
     \filehook@atbegin
     \@@input\@filef@und
     \filehook@parsefile{#1}%
     \filehook@atend
     \filehook@every@atend
    }%
}
%    \end{macrocode}
% \end{macro}
%
% \begin{macro}{\filehook@@default@InputIfFileExists}
%    \begin{macrocode}
\long\gdef\filehook@@default@InputIfFileExists#1#2{%
  \let\InputIfFileExists\filehook@InputIfFileExists
  \IfFileExists{#1}%
    {#2\@addtofilelist{#1}%
     \filehook@parsefile{#1}%
     \filehook@atbegin
     \@@input\@filef@und
     \filehook@parsefile{#1}%
     \filehook@atend
    }%
}
%    \end{macrocode}
% \end{macro}
%
% \begin{macro}{\memoir@InputIfFileExists}
%    \begin{macrocode}
\long\def\memoir@InputIfFileExists#1#2{%
  \IfFileExists{#1}%
    {#2\@addtofilelist{#1}\m@matbeginf{#1}%
     \@@input \@filef@und
     \m@matendf{#1}%
     \killm@matf{#1}}%
}
%    \end{macrocode}
% \end{macro}
%
% \begin{macro}{\scrlfile@InputIfFileExists}
%    \begin{macrocode}
\long\def\scrlfile@InputIfFileExists#1#2{%
  \begingroup\expandafter\expandafter\expandafter\endgroup
  \expandafter\ifx\csname #1-@alias\endcsname\relax
    \expandafter\@secondoftwo
  \else
    \scr@replacefile@msg{\csname #1-@alias\endcsname}{#1}%
    \expandafter\@firstoftwo
  \fi
  {%
    \expandafter\InputIfFileExists\expandafter{\csname
      #1-@alias\endcsname}{#2}%
  }%
  {\IfFileExists{#1}{%
      \scr@load@hook{before}{#1}%
      #2\@addtofilelist{#1}%
      \@@input \@filef@und
      \scr@load@hook{after}{#1}%
    }}%
}
%    \end{macrocode}
% \end{macro}
%
% \begin{macro}{\filehook@scrlfile@InputIfFileExists}
%    \begin{macrocode}
\long\def\filehook@scrlfile@InputIfFileExists#1#2{%
  \begingroup\expandafter\expandafter\expandafter\endgroup
  \expandafter\ifx\csname #1-@alias\endcsname\relax
    \expandafter\@secondoftwo
  \else
    \scr@replacefile@msg{\csname #1-@alias\endcsname}{#1}%
    \expandafter\@firstoftwo
  \fi
  {%
    \expandafter\InputIfFileExists\expandafter{\csname
      #1-@alias\endcsname}{#2}%
  }%
  {\IfFileExists{#1}{%
      \scr@load@hook{before}{#1}%
      #2\@addtofilelist{#1}%
      \filehook@parsefile{#1}%
      \filehook@every@atbegin
      \filehook@atbegin
      \@@input \@filef@und
      \filehook@parsefile{#1}%
      \filehook@atend
      \filehook@every@atend
      \scr@load@hook{after}{#1}%
    }}%
}
%    \end{macrocode}
% \end{macro}
%
% \begin{macro}{\filehook@@scrlfile@InputIfFileExists}
%    \begin{macrocode}
\long\def\filehook@@scrlfile@InputIfFileExists#1#2{%
  \let\InputIfFileExists\filehook@InputIfFileExists
  \begingroup\expandafter\expandafter\expandafter\endgroup
  \expandafter\ifx\csname #1-@alias\endcsname\relax
    \expandafter\@secondoftwo
  \else
    \scr@replacefile@msg{\csname #1-@alias\endcsname}{#1}%
    \expandafter\@firstoftwo
  \fi
  {%
    \expandafter\InputIfFileExists\expandafter{\csname
      #1-@alias\endcsname}{#2}%
  }%
  {\IfFileExists{#1}{%
      \scr@load@hook{before}{#1}%
      #2\@addtofilelist{#1}%
      \filehook@parsefile{#1}%
      \filehook@atbegin
      \@@input \@filef@und
      \filehook@parsefile{#1}%
      \filehook@atend
      \scr@load@hook{after}{#1}%
    }}%
}
%    \end{macrocode}
% \end{macro}
%
%    \begin{macrocode}
}
%    \end{macrocode}
%
% \begin{macro}{\filehook@alt@InputIfFileExists}
% The default and alternate definitions of \Macro\InputIfFileExists are now
% stored in this macro. It is marked as `onlypreamble', which will also clear
% the macro when it is not needed anymore.
%    \begin{macrocode}
\xdef\filehook@alt@InputIfFileExists{%
    \the\toks@
}
\@onlypreamble\filehook@alt@InputIfFileExists
%    \end{macrocode}
% \end{macro}
%
%    \begin{macrocode}
\endgroup
%    \end{macrocode}
%
% \begin{macro}{\filehook@if@scrlfile}
% If the |scrlfile| package definition is detected the |filehook|s are added
% to that definition. Unfortunately the \Macro\scr@load@hook{before} hook is placed \emph{before}
% not after the |#2\@addtofilelist{#1}| code. Otherwise the |filehook|s could simply be added to these hooks.
% Note that this will stop working if |scrlfile| ever changes its definition of the \Macro\InputIfFileExists macro.
%    \begin{macrocode}
\def\filehook@if@scrlfile#1{%
  \@ifpackageloaded {scrlfile}{%
  \ifx\InputIfFileExists\scrlfile@InputIfFileExists
    \global\let\filehook@InputIfFileExists\filehook@scrlfile@InputIfFileExists
    \global\let\filehook@@InputIfFileExists\filehook@@scrlfile@InputIfFileExists
    \global\let\InputIfFileExists\filehook@InputIfFileExists
    \PackageInfo{filehook}{Package 'scrlfile' detected and compensated for}%
  \else
    \iffilehook@force
      \global\let\filehook@InputIfFileExists\filehook@default@InputIfFileExists
      \global\let\filehook@@InputIfFileExists\filehook@@default@InputIfFileExists
      \global\let\InputIfFileExists\filehook@InputIfFileExists
      \PackageWarning{filehook}{Detected 'scrlfile' package with unknown definition of \string\InputIfFileExists.^^J%
                                The 'force' option of 'filehook' is in effect. Macro is overwritten with default!}%
    \else
      \PackageError{filehook}{Detected 'scrlfile' package with unknown definition of \string\InputIfFileExists.^^J%
                              Use the 'force' option of 'filehook' to overwrite it.}%
    \fi
  \fi
  #1}%
}
\@onlypreamble\filehook@if@scrlfile
%    \end{macrocode}
% \end{macro}
%
% \begin{macro}{\filehook@if@memoir}
%    \begin{macrocode}
\def\filehook@if@memoir#1{%
  \@ifclassloaded {memoir}{%
  \ifx\InputIfFileExists\memoir@InputIfFileExists
    \global\let\filehook@InputIfFileExists\filehook@default@InputIfFileExists
    \global\let\filehook@@InputIfFileExists\filehook@@default@InputIfFileExists
    \global\let\InputIfFileExists\filehook@InputIfFileExists
    \AtBeginOfFiles{\expandafter\m@matbeginf\expandafter{\filehookfilearg}}%
    \AtEndOfFiles{\expandafter\m@matendf\expandafter{\filehookfilearg}%
                  \expandafter\killm@matf\expandafter{\filehookfilearg}}%
    \PackageInfo{filehook}{Detected 'memoir' class: the memoir hooks will be moved to the 'At...OfFiles' hooks}
  \else
    \iffilehook@force
      \global\let\filehook@InputIfFileExists\filehook@default@InputIfFileExists
      \global\let\filehook@@InputIfFileExists\filehook@@default@InputIfFileExists
      \global\let\InputIfFileExists\filehook@InputIfFileExists
      \PackageWarning{filehook}{Detected 'memoir' class with unknown definition of \string\InputIfFileExists.^^J%
                                The 'force' option of 'filehook' is in effect. Macro is overwritten with default!}%
    \else
      \PackageError{filehook}{Detected 'memoir' class with unknown definition of \string\InputIfFileExists.^^J%
                              Use the 'force' option of 'filehook' to overwrite it.}%
    \fi
  \fi
  #1}%
}
\@onlypreamble\filehook@if@memoir
%    \end{macrocode}
% \end{macro}
%
% \begin{macro}{\InputIfFileExists}
% The default and alternative definitions of \Macro\InputIfFileExists are first expanded inside a group.
%    \begin{macrocode}
\begingroup
\filehook@alt@InputIfFileExists
%    \end{macrocode}
% First we test for the |scrlfile| package. The test macro adds the necessary patches if so.
% In order to also support it when it is loaded afterwards the two hooks below are used to revert the definition
% before the package and patch it afterwards.
%    \begin{macrocode}
\filehook@if@scrlfile{}{%
  \AtBeginOfPackageFile{scrlfile}{%
    {\filehook@alt@InputIfFileExists
     \global\let\InputIfFileExists\latex@InputIfFileExists
    }%
  }%
  \AtEndOfPackageFile{scrlfile}{%
    {\filehook@alt@InputIfFileExists
    \filehook@if@scrlfile{}{}}%
  }%
%    \end{macrocode}
% If |memoir| is detected its hooks
% are added to the appropriate |At...OfFiles| hooks. This works fine because its hooks have the
% exact same position. Please note that the case when |memoir| is used together with |scrlfile| is not explicitly covered.
% In this case the |scrlfile| package will overwrite |memoir|s definition.
%    \begin{macrocode}
  \filehook@if@memoir{}{%
    \ifx\documentclass\@twoclasseserror\else
      \AtBeginOfClassFile{memoir}{%
        {\filehook@alt@InputIfFileExists
         \global\let\filehook@@InputIfFileExists\latex@InputIfFileExists
         \global\let\InputIfFileExists\latex@InputIfFileExists
         \global\let\@iinput\filehook@orig@@iinput
        }%
      }%
      \AtEndOfClassFile{memoir}{%
        \global\let\@iinput\filehook@@iinput
        {\filehook@alt@InputIfFileExists
        \filehook@if@memoir{}{}}%
      }%
    \fi
%    \end{macrocode}
% Finally, if no specific alternate definition is detected the original \LaTeX\ definition is checked for and a
% error is given if any other unknown definition is detected.
% The \opt{force} option will change the error into a warning and overwrite the macro with the default.
%    \begin{macrocode}
    \ifx\InputIfFileExists\latex@InputIfFileExists
      \global\let\filehook@InputIfFileExists\filehook@default@InputIfFileExists
      \global\let\filehook@@InputIfFileExists\filehook@@default@InputIfFileExists
      \global\let\InputIfFileExists\filehook@InputIfFileExists
    \else
      \iffilehook@force
        \global\let\filehook@InputIfFileExists\filehook@default@InputIfFileExists
        \global\let\filehook@@InputIfFileExists\filehook@@default@InputIfFileExists
        \global\let\InputIfFileExists\filehook@InputIfFileExists
        \PackageWarning{filehook}{Detected unknown definition of \string\InputIfFileExists.^^J%
                                  The 'force' option of 'filehook' is in effect. Macro is overwritten with default!}%
      \else
        \PackageError{filehook}{Detected unknown definition of \string\InputIfFileExists.^^J%
                                Use the 'force' option of 'filehook' to overwrite it.}%
      \fi
    \fi
}}
\endgroup
%    \end{macrocode}
% \end{macro}
%
%    \begin{macrocode}
\AtBeginDocument{%
    \ifx\InputIfFileExists\filehook@InputIfFileExists\else
        \PackageWarning{filehook}{Macro \string\InputIfFileExists\space got redefined after 'filehook' was loaded.^^J%
                                  Certain file hooks might now be dysfunctional!}
    \fi
}
%    \end{macrocode}
%
% \iffalse
%</package>
% \fi
% \Finale
\endinput
