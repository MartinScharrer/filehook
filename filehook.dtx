% \iffalse meta-comment
%% Copyright (c) 2010 by Martin Scharrer <martin@scharrer-online.de>
%% -----------------------------------------------------------------
%%
%% This work may be distributed and/or modified under the
%% conditions of the LaTeX Project Public License, either version 1.3c
%% of this license or (at your option) any later version.
%% The latest version of this license is in
%%
%%   http://www.latex-project.org/lppl.txt
%%
%% and version 1.3c or later is part of all distributions of LaTeX
%% version 2008/05/04 or later.
%%
%% This work has the LPPL maintenance status `maintained'.
%%
%% The Current Maintainer of this work is Martin Scharrer.
%%
%% This work consists of the files filehook.dtx, filehook.ins
%% and the derived file filehook.sty.
%%
%% $Id: standalone.dtx 1803 2010-04-03 13:59:44Z martin $
% \fi
%
% \iffalse
%<*driver>
\ProvidesFile{filehook.dtx}
%</driver>
%<package>\NeedsTeXFormat{LaTeX2e}[1999/12/01]
%<package>\ProvidesPackage{filehook}
%<*package>
    [2010/12/06 v0.2 Hooks for input files]
%</package>
%
%<*driver>
\documentclass{ltxdoc}
\usepackage{filehook}[2010/12/06]
\usepackage{ifpdf}
\usepackage{hyperref}
\EnableCrossrefs
\CodelineIndex
\RecordChanges
\providecommand*\pkg{\texttt}
\begin{document}
  \DocInput{filehook.dtx}
  \PrintChanges
  \PrintIndex
\end{document}
%</driver>
% \fi
%
% \CheckSum{251}
%
% \CharacterTable
%  {Upper-case    \A\B\C\D\E\F\G\H\I\J\K\L\M\N\O\P\Q\R\S\T\U\V\W\X\Y\Z
%   Lower-case    \a\b\c\d\e\f\g\h\i\j\k\l\m\n\o\p\q\r\s\t\u\v\w\x\y\z
%   Digits        \0\1\2\3\4\5\6\7\8\9
%   Exclamation   \!     Double quote  \"     Hash (number) \#
%   Dollar        \$     Percent       \%     Ampersand     \&
%   Acute accent  \'     Left paren    \(     Right paren   \)
%   Asterisk      \*     Plus          \+     Comma         \,
%   Minus         \-     Point         \.     Solidus       \/
%   Colon         \:     Semicolon     \;     Less than     \<
%   Equals        \=     Greater than  \>     Question mark \?
%   Commercial at \@     Left bracket  \[     Backslash     \\
%   Right bracket \]     Circumflex    \^     Underscore    \_
%   Grave accent  \`     Left brace    \{     Vertical bar  \|
%   Right brace   \}     Tilde         \~}
%
%
% \changes{v0.1}{2010/04/08}{Initial version}
%
% \GetFileInfo{filehook.dtx}
%
% \DoNotIndex{\newcommand,\newenvironment}
% 
% \ifpdf
% \hypersetup{%
%   pdfauthor   = {Martin Scharrer <martin@scharrer-online.de>},
%   pdftitle    = {The filehook package},
%   pdfsubject  = {Documentation of LaTeX package 'filehook'},
%   pdfkeywords = {current filename, filename, file, name, LaTeX, TeX}
% }%
% \fi
% \clearpage
% \null
% \vspace*{-2em}
% \begin{center}
%   {\LARGE\sffamily The \emph{filehook} Package\\[\medskipamount]}
%   {\large Martin Scharrer \\[\medskipamount]\normalsize 
%   \url{martin@scharrer-online.de}\\[.8ex]
%   \url{http://www.ctan.org/pkg/standalone/}\\[\bigskipamount]}
%   {\large Version \fileversion\ -- \filedate}\\
% \end{center}
% \vspace{1.2em}%
%
% \begin{abstract}
% This small package provides hooks for input files. Document and package authors can use these hooks to
% execute code at begin or the end of specific or all input files.
% \end{abstract}
%
% \section{Introduction}
% These package changes some internal \LaTeX{} macros used to load input files so that they include `hooks'.
% A hook is an (internal) macro executed at specific points. Normally it is initially empty, but can be extended using
% an user level macro. The most common hook in \LaTeX{} is the `At-Begin-Document' hook. Code can be added to this hook
% using \cs{AtBeginDocument}\marg{code}.
%
% \section{Usage}
% This package provides three groups of hooks: for file read using |\input|, for files read using |\include| and for all read files (i.e.\ all files read using
% |\InputIfFileExists|, which includes package and class files and files falling into the first two groups).
% All groups include a `AtBegin' and a `AtEnd' macro. The |\include| group has also a `After' hook which
% it is executed \emph{after} the page break (|\clearpage|) is inserted by the |\include| code. In contrast, the `AtEnd' hook is executed before the trailing page break 
% and the `AtBegin' hook is executed after the \emph{leading} page break.
%
% Each group includes general and file specific hooks. The general hooks are executed for every file of this group and provide the file name as argument |#1|.
% The file specific ones are only executed for a certain file. 
%
% The below macros can be used to add material (\TeX{} code) to the related hooks. All `AtBegin' macros will \emph{append} the code to the hooks, but the
% `AtEnd' and `After' macros will \emph{prefix} the code instead. This ensures that two different packages adding material in `AtBegin'/`AtEnd' pairs do not
% overlap each other. Instead the later used package adds the code closer to the file content, `inside' the material added by the first package.
% Therefore it is safely possible to surround the content of a file with multiple \LaTeX{} environments using multiple `AtBegin'/`AtEnd' macro calls.
% If required inside another package a different order can be enforced by using the internal hook macros shown in the implementation section.
%
% \subsection*{Include Files}
% \DescribeMacro{\AtBeginOfIncludes}
% \DescribeMacro{\AtEndOfIncludes}
% \DescribeMacro{\AfterIncludes}
% All these macro take one argument (some \TeX{} code) which is added to the specific hook for files read using |\include|. The code can use the macro argument |#1|
% which will be expanded to the include file name, i.e.\ the hooks are macros with one argument which will be the file name.
% As described above the `AtEnd' hook is executed before and the `After' hook is executed after the trailing |\clearpage|.
% Material which should be (still) valid in the page header or footer of the last page of such an
% file should therefore use the `After' hook. 
%
% \DescribeMacro{\AtBeginOfIncludeFile}
% \DescribeMacro{\AtEndOfIncludeFile}
% \DescribeMacro{\AfterIncludeFile}
% These file-specific macros take the two arguments \marg{file name}\marg{code}. The \meta{code} is only executed for the file with the given \meta{file name}
% and only if it is read using |\include|. It is not allowed to use macro arguments inside the code.
% The \meta{file name} should be identical to the name used for |\include| and not include the `|.tex|' extension.
%
% \subsection*{Input Files}
%
% \DescribeMacro{\AtBeginOfInputs}
% \DescribeMacro{\AtEndOfInputs}
% Like the |\...OfIncludes|\marg{code} macros above, just for file read using |\input|. Again, the macro argument |#1| can be used inside the \meta{code} and
% will be expanded to the file name.
%
% \DescribeMacro{\AtBeginOfInputFile}
% \DescribeMacro{\AtEndOfInputFile}
% Like the |\...OfIncludeFile|\marg{file name}\marg{code} macros above, just for file read using |\input|. Here the \meta{file name} should include the file
% extension! The \meta{code} must not include any macro arguments (|#1|).
%
% \subsection*{All Files}
%
% \DescribeMacro{\AtBeginOfFiles}
% \DescribeMacro{\AtEndOfFiles}
% These macros add the given \marg{code} to two hooks executed for all files read using the |\InputIfFileExists| macro. This macro is used internally by the 
% |\input|, |\include| and |\usepackage|/|\RequirePackage| macros. Packages and classes might use it to include additional or auxiliary files.
% Authors can exclude those files from the hooks by using |\IfFileExists|\marg{file name}|{\@input\@filef@und}{}| instead.
%
% \DescribeMacro{\AtBeginOfFile}
% \DescribeMacro{\AtEndOfFile}
% Like the |\...OfIncludeFile|\marg{file name}\marg{code} macros above, just for `all' read files. Here the \meta{file name} should include the file
% extension! The \meta{code} must not include any macro arguments (|#1|).
%
%
% The `all files' hooks are closer to the file content than the |\input| and |\include| hook, i.e.\ the \cs{AtBeginOfFiles} comes after the \cs{AtBeginOfIncludes} and
% the \cs{AtEndOfFiles} comes before the \cs{AtEndOfIncludes} hook.
%
% \StopEventually{}
%
% \section{Compatibility Issues with other Packages}
% This sections lists packages and classes which might cause issues when used together with |filehook|. In general all code which also redefines |\InputIfFileExists|
% will cause trouble. Note that at the moment |filehook| simply redefines this macro to the original \LaTeX\ definition plus added hooks and any previous modifications
% will be lost.
%
% \subsection*{listings}
% The \pkg{listings} package uses |\input| inside \cs{lstinputlisting}. Therefore the |InputFile|(|s|) and |File|(|s|) hooks are also triggered for these files.
% Please note that this hooks are executing inside a verbatim environment. While the code in the hook is not affected (because it was added outside the verbatim
% environment), any further code read using any input macro (|\input|, |\@input|, |\@@input| (\TeX's |\input|), \ldots) will be processed verbatim and typeset
% as part of the listing. A known package suffering from this is \pkg{svn-multi} which loads |.svx| files for every |.tex| file.
%
% \subsection*{fink}
% The |filehook| and |currfile| packages where written as replacements for the |fink| package, where |filehook| provides the necessary hooks for |currfile|.
% The |fink| package has now been deprecated in favour of |currfile| and should not be used anymore. The |fink| compatibility code has been removed from |filehook|
% and both cannot be used successfully together as both redefine the |\InputIfFileExists| macro.
%
% \subsection*{jmlrbook}
% The |jmlrbook| class from the |jmlr| bundle temporary redefines |\InputIfFileExists| to import papers.
% The `original' definition is saved away at load time of the package and is used internally by the new definition.
% This means that the hooks will not be active for this imported files because |filehook| is loaded after the class.
% This should not affect the normal usage.
% Note that, in theory, the package could be loaded before |\documentclass| using |\RequirePackage| to enable the file hooks also for these files.
%
% \subsection*{scrlfile}
% The |scrlfile| package from the \emph{koma-script} bundle redefines |\InputIfFileExists| to allow file name aliases and to also add hooks.
% It checks for the original \LaTeX\ definition of this macro and will trigger an error if it was changed beforehand.
% At the moment both packages will not work together.
%
% \section{Implementation}\label{sec:impl}
%
% \iffalse
%<*package>
% \fi
%
% \subsection{Installation of Hooks}
% The \cs{@input@} and \cs{@iinput} macros from |latex.ltx| are redefined to install the hooks.
%
% First the original definitions are saved away.
%    \begin{macrocode}
\let\filehook@orig@@input@\@input@
\let\filehook@orig@@iinput\@iinput
%    \end{macrocode}
%
% \begin{macro}{\@input@}
% This macro is redefined for the |\include| file hooks.
% Checks if the next command is |\clearpage| which indicates that we are inside \cs{@include}.
% If so the hooks are installed, otherwise the original macro is used unchanged.
% For the `after' hook an own |\clearpage| is inserted and the original one is gobbled.
%
%    \begin{macrocode}
\def\@input@#1{%
  \@ifnextchar\clearpage
    {\filehook@include@atbegin{#1}%
     \filehook@orig@@input@{#1}%
     \filehook@include@atend{#1}%
     \clearpage
     \filehook@include@after{#1}%
     \@gobble
    }%
    {\filehook@orig@@input@{#1}}%
}
%    \end{macrocode}
% \end{macro}
%
% \begin{macro}{\@iinput}
% This macro is redefined for the |\input| file hooks.
% it simply surrounds the original macro with the hooks.
%    \begin{macrocode}
\def\@iinput#1{%
  \filehook@input@atbegin{#1}%
  \filehook@orig@@iinput{#1}%
  \filehook@input@atend{#1}%
}
%    \end{macrocode}
% \end{macro}
%
% \begin{macro}{\InputIfFileExists}
% This macro is redefined for the general file hooks.
% The original definition is not saved away and called by the new definition, because
% of the existing complexity. The hooks must be places around the actual input macro (|\@@input|).
%
%    \begin{macrocode}
\long\def\InputIfFileExists#1#2{%
  \IfFileExists{#1}%
    {#2\@addtofilelist{#1}%
     \filehook@atbegin{#1}%
     \@@input\@filef@und
     \filehook@atend{#1}%
    }%
}
%    \end{macrocode}
% \end{macro}
%
%
% \subsection{Initialisation of Hooks}
% The general hooks are initialised to call the file specific hooks.
%
% \begin{macro}{\filehook@include@atbegin}
% \begin{macro}{\filehook@include@atend}
% \begin{macro}{\filehook@include@after}
%    \begin{macrocode}
\def\filehook@include@atbegin#1{%
  \@nameuse{\filehook@include@atbegin@#1}%
}
\def\filehook@include@atend#1{%
  \@nameuse{\filehook@include@atend@#1}%
}
\def\filehook@include@after#1{%
  \@nameuse{\filehook@include@after@#1}%
}
%    \end{macrocode}
% \end{macro}
% \end{macro}
% \end{macro}
%
% \begin{macro}{\filehook@input@atbegin}
% \begin{macro}{\filehook@input@atend}
%    \begin{macrocode}
\def\filehook@input@atbegin#1{%
  \@nameuse{\filehook@input@atbegin@#1}%
}
\def\filehook@input@atend#1{%
  \@nameuse{\filehook@input@atend@#1}%
}
%    \end{macrocode}
% \end{macro}
% \end{macro}
%
% \begin{macro}{\filehook@atbegin}
% \begin{macro}{\filehook@atend}
%    \begin{macrocode}
\def\filehook@atbegin#1{%
  \@nameuse{\filehook@atbegin@#1}%
}
\def\filehook@atend#1{%
  \@nameuse{\filehook@atend@#1}%
}
%    \end{macrocode}
% \end{macro}
% \end{macro}
%
%
% \subsection{Hook Modification Macros}
% The following macros are used to modify the hooks, i.e.\ to prefix or append code to them.
%
% \subsubsection*{Internal Macros}
%
% The macro prefixes for the file specific hooks are stored in macros to reduce the number of 
% tokens in the following macro definitions.
%    \begin{macrocode}
\def\filehook@include@atbegin@{filehook@include@atbegin@}
\def\filehook@include@atend@{filehook@include@atend@}
\def\filehook@include@after@{filehook@include@after@}
\def\filehook@input@atbegin@{filehook@input@atbegin@}
\def\filehook@input@atend@{filehook@input@atend@}
\def\filehook@input@after@{filehook@input@after@}
\def\filehook@atbegin@{filehook@atbegin@}
\def\filehook@atend@{filehook@atend@}
\def\filehook@after@{filehook@after@}
%    \end{macrocode}
%
%
% \begin{macro}{\filehook@append}
% Uses default \LaTeX{} macro.
%    \begin{macrocode}
\def\filehook@append{\g@addto@macro}
%    \end{macrocode}
% \end{macro}
%
% \begin{macro}{\filehook@appendwarg}
% Appends code with one macro argument. The |\@tempa| intermediate step is required because of the
% included |##1| which wouldn't correctly expand otherwise.
%    \begin{macrocode}
\long\def\filehook@appendwarg#1#2{%
  \begingroup
    \toks@\expandafter{#1{##1}#2}%
    \edef\@tempa{\the\toks@}%
    \expandafter\gdef\expandafter#1\expandafter##\expandafter1\expandafter{\@tempa}%
  \endgroup
}
%    \end{macrocode}
% \end{macro}
%
%
% \begin{macro}{\filehook@prefix}
% Prefixes code without an argument to a hook.
%    \begin{macrocode}
\long\def\filehook@prefix#1#2{%
  \begingroup
    \@temptokena{#2}%
    \toks@\expandafter{#1}%
    \xdef#1{\the\@temptokena\the\toks@}%
  \endgroup
}
%    \end{macrocode}
% \end{macro}
%
% \begin{macro}{\filehook@prefixwarg}
% Prefixes code with an argument to a hook.
%    \begin{macrocode}
\long\def\filehook@prefixwarg#1#2{%
  \begingroup
    \@temptokena{#2}%
    \toks@\expandafter{#1{##1}}%
    \edef\@tempa{\the\@temptokena\the\toks@}%
    \expandafter\gdef\expandafter#1\expandafter##\expandafter1\expandafter{\@tempa}%
  \endgroup
}
%    \end{macrocode}
% \end{macro}
%
%
% \subsubsection*{User Level Macros}
% The user level macros simple use the above defined macros on the appropriate hook.
%
% \begin{macro}{\AtBeginOfIncludes}
%    \begin{macrocode}
\newcommand*\AtBeginOfIncludes{%
  \filehook@appendwarg\filehook@include@atbegin
}
%    \end{macrocode}
% \end{macro}
%
% \begin{macro}{\AtEndOfIncludes}
%    \begin{macrocode}
\newcommand*\AtEndOfIncludes{%
  \filehook@prefixwarg\filehook@include@atend
}
%    \end{macrocode}
% \end{macro}
%
% \begin{macro}{\AfterOfIncludes}
%    \begin{macrocode}
\newcommand*\AfterIncludes{%
  \filehook@prefixwarg\filehook@include@after
}
%    \end{macrocode}
% \end{macro}
%
% \begin{macro}{\AtBeginOfIncludeFile}
%    \begin{macrocode}
\newcommand*\AtBeginOfIncludeFile[1]{%
  \@ifundefined{\filehook@include@atbegin@#1.tex}%
    {\long\global\@namedef{\filehook@include@atbegin@#1.tex}}%
    {\expandafter\filehook@append\csname\filehook@include@atbegin@#1.tex\endcsname}%
}
%    \end{macrocode}
% \end{macro}
%
% \begin{macro}{\AtEndOfIncludeFile}
%    \begin{macrocode}
\newcommand*\AtEndOfIncludeFile[1]{%
  \@ifundefined{\filehook@include@atend@#1.tex}%
    {\long\global\@namedef{\filehook@include@atend@#1.tex}}%
    {\expandafter\filehook@prefix\csname\filehook@include@atend@#1.tex\endcsname}%
}
%    \end{macrocode}
% \end{macro}
%
% \begin{macro}{\AfterOfIncludeFile}
%    \begin{macrocode}
\newcommand*\AfterOfIncludeFile[1]{%
  \@ifundefined{\filehook@include@after@#1.tex}%
    {\long\global\@namedef{\filehook@include@after@#1.tex}}%
    {\expandafter\filehook@prefix\csname\filehook@include@after@#1.tex\endcsname}%
}
%    \end{macrocode}
% \end{macro}
%
% \begin{macro}{\AtBeginOfInputs}
%    \begin{macrocode}
\newcommand*\AtBeginOfInputs{%
  \filehook@appendwarg\filehook@input@atbegin
}
%    \end{macrocode}
% \end{macro}
%
% \begin{macro}{\AtEndOfInputs}
%    \begin{macrocode}
\newcommand*\AtEndOfInputs{%
  \filehook@prefixwarg\filehook@input@atend
}
%    \end{macrocode}
% \end{macro}
%
% \begin{macro}{\AtBeginOfInputFile}
%    \begin{macrocode}
\newcommand*\AtBeginOfInputFile[1]{%
  \@ifundefined{\filehook@input@atbegin@#1}%
    {\long\global\@namedef{\filehook@input@atbegin@#1}}%
    {\expandafter\filehook@append\csname\filehook@input@atbegin@#1\endcsname}%
}
%    \end{macrocode}
% \end{macro}
%
% \begin{macro}{\AtEndOfInputFile}
%    \begin{macrocode}
\newcommand*\AtEndOfInputFile[1]{%
  \@ifundefined{\filehook@input@atend@#1}%
    {\long\global\@namedef{\filehook@input@atend@#1}}%
    {\expandafter\filehook@prefix\csname\filehook@input@atend@#1\endcsname}%
}
%    \end{macrocode}
% \end{macro}
%
% \begin{macro}{\AtBeginOfFiles}
%    \begin{macrocode}
\newcommand*\AtBeginOfFiles{%
  \filehook@appendwarg\filehook@atbegin
}
%    \end{macrocode}
% \end{macro}
%
% \begin{macro}{\AtEndOfFiles}
%    \begin{macrocode}
\newcommand*\AtEndOfFiles{%
  \filehook@prefixwarg\filehook@atend
}
%    \end{macrocode}
% \end{macro}
%
% \begin{macro}{\AtBeginOfFile}
%    \begin{macrocode}
\newcommand*\AtBeginOfFile[1]{%
  \@ifundefined{\filehook@atbegin@#1}%
    {\long\global\@namedef{\filehook@atbegin@#1}}%
    {\expandafter\filehook@append\csname\filehook@atbegin@#1\endcsname}%
}
%    \end{macrocode}
% \end{macro}
%
% \begin{macro}{\AtEndOfFile}
%    \begin{macrocode}
\newcommand*\AtEndOfFile[1]{%
  \@ifundefined{\filehook@atend@#1}%
    {\long\global\@namedef{\filehook@atend@#1}}%
    {\expandafter\filehook@prefix\csname\filehook@atend@#1\endcsname}%
}
%    \end{macrocode}
% \end{macro}
%
%
% \iffalse
%</package>
% \fi
% \Finale
\endinput
